\documentclass[../LabBook]{subfiles}

% Document
\begin{document}
\section{PowerShell Commands}

    Open windows explorer from this directory. ii is short for Invoke-Item
    \begin{minted}{powershell}
        ii .
    \end{minted}



\section{DLLs}

\subsection{Creating a DLL and loading functions from it}

Here's a little library that can be compiled as a dll:

    \begin{minted}{c++}
        // Hello.cpp
        extern "C" char const * __cdecl GetGreeting()
            {
                return "Hello, C++ Programmers!";
            }
    \end{minted}

    \begin{minted}{c++}
        // PrintGreeting.cpp
        #include <stdio.h>
        #include <Windows.h>

        int main(){
            HMODULE const HelloDll = LoadLibraryExW(L"test.dll", nullptr, 0);

            /*
             * GetGreetingType is a function pointer for the type we want to load from Hello.dll
             */
            using GetGreetingType = char const* (__cdecl*)();

            // then we load get greeting, casting to the type we loaded.
            GetGreetingType const GetGreeting = reinterpret_cast<GetGreetingType>(
                GetProcAddress(
                    HelloDll, "GetGreeting"));

            puts(GetGreeting());

            FreeLibrary(HelloDll);
        }
    \end{minted}

    You can compile this using visual studio developer command prompt. The /c flag tells cl only to compile and not also link Hello.cpp

    \begin{minted}{cmd}
        > cl.exe /c Hello.cpp
    \end{minted}

    We have just created Hello.obj. Now we can link into a dll:

    \begin{minted}{cmd}
        > link.exe Hello.obj /DLL /NOENTRY /EXPORT:GetGreeting
    \end{minted}

    The DLL flag specifies to create a DLL. The NOENTRY flag tells the linker that the dll
    does not have an entry point and the /EXPORT:GetGreeting tells the linker which functions from the DLL
    are going to be exported into another library.






\end{document}